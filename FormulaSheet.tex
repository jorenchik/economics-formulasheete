\documentclass[9pt]{article}

\usepackage{sectsty}
\usepackage{graphicx}
\usepackage{amsmath}
\usepackage{multicol}
% \pagestyle{empty}

% Margins
\topmargin=-0.45in
\evensidemargin=0in
\oddsidemargin=0in
\textwidth=6.5in
\textheight=9.0in
\headsep=0.25in

% \title{ PD2 Formulu lapa}
% \author{ Jorens Štekeļs}
% \date{\today}

\begin{document}

\begin{multicols}{2}
    \begin{center}
        \textbf{IKP}

        \begin{falign*}
            PV &= P_p - P_{sp} \\ 
            \intertext{PV - pievienotā vērtība;} \\
            \intertext{$P_p$ - produkta vērtība;}\\
            \intertext{$P_{sp}$ - starp produktu vērtība.}\\
        \end{falign*}
        \vspace*{1em}

        \begin{falign*}
            IKP &= PV + N_n \\ 
            \intertext{PV - pievienotā vērtība;} \\
            \intertext{$N_n$ - netiešie nodokļi;}\\
            \intertext{IKP - ienākumu aspekts.}\\
        \end{falign*}
        \vspace*{1em}
        \begin{falign*}
            IKP &= C + G + Ig + X_n \\ 
            \intertext{ C - personīgās izmaksas;} \\
            \intertext{ G - valsts izmaksas;}\\
            \intertext{ Ig - kopējās investīcijas;}\\
            \intertext{ $X_n$ - tīrais eksports;}\\
            \intertext{ IKP - izdevumu aspekts;}\\
        \end{falign*}
        \vspace*{1em}
        \begin{falign*}
            X_n &= X - M \\ 
            \intertext{ X - eksports;} \\
            \intertext{ M - imports;}\\
            \intertext{ $X_n$ - tīrais eksports.}\\
        \end{falign*}
        \vspace*{1em}

        \begin{falign*}
            IKP &= S + (T_{pi} - SUB) + TRE + MI \\ 
            \intertext{ S - atlīdzība nodarbinātajiem;} \\
            \intertext{ $T_{pi}$ - ražošanas un importa nodokļi;}\\
            \intertext{ SUB - subsīdijas;}\\
            \intertext{ TRE - kopējs darbības rezultāts;}\\
            \intertext{ MI - jauktais kopienākums;}\\
            \intertext{ IKP - ienākumu aspekts.}\\
        \end{falign*}
        \vspace*{1em}

        \textbf{Nominālais un Reālais IKP}
    
        \begin{falign*}
            RIKP &= L * LP \\
            \vspace*{0.5em}
            \intertext{RIKP - reālais IKP.}\\
            \intertext{L - ieguldīts darbs.}\\
            \intertext{LP - darba ražīgums.}\\
        \end{falign*}
        \vspace*{1em}

        \begin{falign*}
            PCI &= \displaystyle\frac{CB_{rev}}{CB_{ref}} * 100 \\
            \vspace*{0.5em}
            \intertext{$CB_ {rev}$ - Cenu groza vērtība pārskata gadā.}\\
            \intertext{$CB_ {ref}$ - Cenu groza vērtība bāzes gadā.}\\
            \intertext{PCI - patēriņa cenu indekss.}\\
        \end{falign*}
        \vspace*{1em}

        \begin{falign*}
            PCI (I_L) &= \displaystyle\frac{\Sum p_i^t p_i^o}{\Sum p_i^o p_i^o} * 100 \\ 
            \vspace*{0.5em}
            \intertext{t - pārskata periods.}\\
            \intertext{o - bāzes periods.}\\
            \intertext{$PCI (I_l)$ - Lāspera cenu indekss.}\\
        \end{falign*}
        \vspace*{1em}

        \begin{falign*}
            IKP_d &= \displaystyle\frac{IKP}{RINP} * 100 \\ 
            \vspace*{0.5em}
            \intertext{$IKP_d$ - IKP deflators} \\
        \end{falign*}
        \vspace*{1em}

        \begin{falign*}
            RIKP_g &= \displaystyle\frac{RIKP_i - RIKP_{i-1}}{RIKP_{i_i}} * 100 \\ 
            \vspace*{0.5em}
            \intertext{$RIKP_g$ - Ikgadējais reālā IKP pieauguma temps. } \\
        \end{falign*}
        \vspace*{1em}

        \begin{falign*}
            INF_r &= \displaystyle\frac{INF_i - INF_{i-1}}{INF_{i_i}} * 100 \\ 
            \vspace*{0.5em}
            \intertext{$INF_g$ - inflācijas temps. } \\
        \end{falign*}
        \vspace*{1em}

        \begin{falign*}
            P &= P_a + P_{na} \\ 
            L_s &= P_e + P_{s} &= P_a \\ 
            E_l &= \displaystyle\frac{P_e}{P_{ea}} \\
            EA_l &= \displaystyle\frac{P_a}{P_{ea}} \\
            UE &= \displaystyle\frac{P_{ue}}{P_{ea}} \\
            \vspace*{0.5em}
            \intertext{$P$ - populācija;} \\
            \intertext{$L_s$ - darba piedāvājums;} \\
            \intertext{$P_e$ - nodarbinātie;} \\
            \intertext{$P_{na}$ - nenodarbinātie;} \\
            \intertext{$P_s$ - darba piedāvājums;} \\
            \intertext{$P_a$ - ekonomiski aktīvi;} \\
            \intertext{$P_{ea}$ - darbspējīga vecuma;} \\
            \intertext{$P_{ue}$ - bezdarbnieki (meklējošie);} \\
            \intertext{$E_l$ - nodarbinātības līmenis;} \\
            \intertext{$EA_l$ - ekonomiskās aktivitātes līmenis;} \\
            \intertext{$UE$ - bedarba līmenis;} \\
        \end{falign*}
        \vspace*{1em}


        \textbf{Nodarbinātība}

        \vspace*{1em}
        \begin{falign*}
            \displaystyle\frac{Y-Y^*}{Y^*} &= -b (u-u^*) \\
            \vspace*{0.5em}
            \intertext{Y - faktiskais IKP;}\\
            \intertext{$Y^*$ - potentiālais IKP;}\\
            \intertext{u - faktiskais bezdarba līmenis;}\\
            \intertext{$u^*$ - dabiskais bezdarba līmenis;}\\
            \intertext{b - empīriskais IKP jūtīguma koeficients cikliskā bezdarba dinamikā;}\\
        \end{falign*}
        \vspace*{1em}


        \begin{falign*}
            IKP + I_o - S_o &= NKI \\
            \vspace*{0.5em}
            \intertext{$I_o$ - no ārvalstīm saņemtie īpašuma ieņēmumi un
            atlīdzība nodarbinātiem}\\
            \intertext{$S_o$ - ārvalstīm samaksāti īpašuma ieņēmumi un
            atlīdzība nodarbinātiem}\\
        \end{falign*}
        \vspace*{1em}

        \begin{falign*}
            IKP + I_o - S_o &= NKI \\
            \vspace*{0.5em}
            \intertext{$I_o$ - no ārvalstīm saņemtie īpašuma ieņēmumi un
            atlīdzība nodarbinātiem}\\
            \intertext{$S_o$ - ārvalstīm samaksāti īpašuma ieņēmumi un
            atlīdzība nodarbinātiem}\\
        \end{falign*}
        \vspace*{1em}

        \begin{falign*}
            IKP &= A + TIKP\\
            \vspace*{0.5em}
            \intertext{A - amortizācija;}\\
            \intertext{TIKP - tīrais IKP.}\\
        \end{falign*}
        \vspace*{1em}

        \begin{falign*}
            NKI &= A + NKI\\
            \vspace*{0.5em}
            \intertext{A - amortizācija;}\\
            \intertext{TNKI - tīrais NKI.}\\
        \end{falign*}
        \vspace*{1em}

        \begin{falign*}
            DI &= NKI - T_u + SUB_{u} - A - P_n - T_p + TR\\
            \vspace*{0.5em}
            \intertext{$T_u$ - nodokļi, ko makstā uzņēmumi}\\
            \intertext{NKI - nacionālkopienākums.}\\
            \intertext{$SUB_u$ - subsīdijas. ko saņem uzņēmēmumi;}\\
            \intertext{A - amortizācija;}\\
            \intertext{$P_n$ - nesadalīta peļņa}\\
            \intertext{$T_p$ - nodokļi, ko maksā personas.}\\
            \intertext{TR - transferti.}\\
            \intertext{DI - rīcībā esošais ienākumi.}\\
        \end{falign*}
        \vspace*{1em}


        \textbf{Bankas}

        \begin{falign*}
            m &= \displaystyle\frac{1}{R} \\
            D &= E \times m \\
            \vspace*{0.5em}
            \intertext{$R$ - obligātās rezerves norma.}\\
            \intertext{m - naudas reizinātājs}\\
            \intertext{E - noguldījumi mīnus obligātās rezerves}\\
            \intertext{D - maksimālais naudas palielinājums}\\
        \end{falign*}
        \vspace*{1em}

    \end{center}
\end{multicols}


\end{document}
